%%% & --translate-file=cp1250pl
%% ************ AKADEMIA GÓRNICZO-HUTNICZA W KRAKOWIE **************
%% ***************** Wydział Matematyki Stosowanej ***************** 
%% ****************** PRACA MAGISTERSKA w LaTeX-u ******************
%%    autor: Tomasz Czyż
%%    Copyright (C) 2003 by ------
%% ************************* Plik główny *************************

%%
%% ======== PREAMBUŁA ======== 
%%
\documentclass[oik, pdftex, robocza, man]{mgrwms}

\usepackage[utf8]{inputenc}  % opcja latin2 dla Linux
\usepackage{amsmath}           % łatwiejszy skład matematyki
\newcommand\numberthis{\addtocounter{equation}{1}\tag{\theequation}}
\usepackage{amssymb} 
\usepackage{latexsym}
\usepackage{amsthm}
\usepackage{enumerate}

\usepackage{color}

\usepackage[polish]{babel}
\usepackage[OT4]{fontenc}
\usepackage{polski}
\allowdisplaybreaks
%% <<<< BiBTeX >>>>
% \bibliographystyle{ddabbrv}
% \nocite{*}

\begin{document}
%%
%% ======== METRYCZKA PRACY ========
%%
\title{ \LARGE Aproksymacja funkcji kawałkami regularnych przy użyciu informacji dokładnej i niedokładnej}
\author{Tomasz Czyż}
\promotor{dr Maciej Goćwin}
\nralbumu{290565}
\maketitle

\slowakluczowe{słowa kluczowe}
\keywords{keywords}
%%
%% ======== MAKRA ========
%%
%-> Miejsce na nasze makra (jedno z wielu ;). Lepszym pomysłem będzie jednak 
%-> umieszczenie ich w osobnym pliku i wczytanie poleceniem \input
%%
\newtheorem{thm}{\indent Twierdzenie}[chapter]
\newtheorem{lemma}[thm]{\indent Lemat}
\newtheorem{cor}[thm]{\indent Wniosek}
\newtheorem{obs}[thm]{\indent Obserwacja}
\newtheorem{uw}[thm]{\indent Uwaga}
\newtheorem{df}[thm]{Definicja}
\newcommand{\E}{\mathbb{E}}
\newcommand{\R}{\mathbb{R}}
\newcommand{\Pra}{\mathbb{Pra}}
\newcommand{\F}{\mathcal{F}}

\makeatletter
\newcommand*{\defeq}{\mathrel{\rlap{%
                     \raisebox{0.3ex}{$\m@th\cdot$}}%
                     \raisebox{-0.3ex}{$\m@th\cdot$}}%
                     =}
\let\c@table\c@figure
\makeatother

%%
%% ======== SPIS TREŚCI ========
%%
\tableofcontents
%%
%% ======== STRESZCZENIE PRACY (POLSKIE) ========
%%

\begin{streszczenie}
    Streszczenie
\end{streszczenie}

%%
%% ======== STRESZCZENIE PRACY (ANGIELSKIE) ========
%%

\begin{abstract}
    Abstract
\end{abstract}

%%
%%
%% ======== GŁÓWNA CZĘŚĆ PRACY ========
%%
%%

%%
%% ==== WSTĘP ====
%%

\begin{wstep}    % ew. \begin{wstep}[Wprowadzenie]
    Celem niniejszej pracy jest analiza zachowania różnych algorytmów aproksymujących funkcje kawałkami gładkie przy użyciu informacji dokładniej i niedokładniej.
    Pierwszy z analizowanych algorytmów został przedstawiony w pracy \cite{PoA}. Rozważane są w niej funkcje klasy $F^{\infty}_r$, o których zakładamy, że zarówno sama funkcja $f: [0,T] \longrightarrow \R$ może być nieciągła, jak i jej pochodna, począwszy od rzędu możliwe większego od pierwszego. Czyli, dla przykładu, $f$ może być dwa razy różniczkowalna na $[0, T]$ i $f^{(3)}(s)$ może nie istnieć w jakimś punkcie $s$. Ponadto, $f$ może mieć skończenie wiele punktów osobliwych; ich ilość i położenie jest nieznane. Dodatkowo, algorytm przedstawiony w \cite{PoA} używa $n$ wartości funkcji w punktach $x_1, \ldots x_n$ jako jedyne dostępne informacje o funkcji $f$, a w przypadku algorytmu adaptacyjnego dopuszczamy, że wybór $x_j$ zależy od $f(x_1), \ldots, f(x_{j-1})$.
    W wymienionej pracy do znalezienia optymalnego algorytmu nieadaptacyjnego i adaptacyjnego w najgorszym przypadku oraz w przypadku asymptotycznym do mierzenia błędu stosowana jest m.in. norma $L^p (1 \leq  p < \infty)$.

    W artykule \cite{UA} rozszerzony jest wynik prac \cite{IaA} i \cite{PoA} poprzez skupienie się na klasie funkcji ciągłych globalnie r-regularnych z co najwyżej jednym punktem osobliwym. W podanej pracy przedstawiony jest algorytm nieadaptacyjny, który asymptotycznie poprawia ograniczenie błędu z \cite{AoP}.

    Ostatni z analizowanych algorytmów pochodzi z pracy \cite{AoP} i jako jedyny z przedstawionych algorytmów uwzględnia zaburzenie danych. W artykule uogólnione zostają rezultaty z \cite{PoA} i \cite{UA} poprzez wprowadzanie informacji niedokładniej oraz założenie, że wykładnik Höldera $\varrho \in (0,1]$. (...) Z tego powodu w pracy \cite{AoP} przedstawiony został nowy algorytm do lokalizacji osobliwości. Co więcej dla $\varrho = 0$, co odpowiada informacji dokładniej, przedstawiony algorytm jest nawet prostszy niż te z \cite{PoA} i \cite{UA}.

    \end{wstep}

%%
%% ==== ROZDZIAŁ 1 ====
%%

\chapter{Złożoność obliczeniowa}

rozdział o informacji zaburzonej itp.

\mgrclosechapter

%%
%% ==== ROZDZIAŁ 2 ====
%%

\chapter{Ograniczenia z dołu}


\mgrclosechapter


%%
%% ==== ROZDZIAŁ 3 ====
%%

\chapter{Algorytmy}


\section{Algorytm oparty o informację zaburzoną}

W tej opiszemy algorytm bazujący na informacji zaburzonej przedstawiony w artykule \cite{AoP}. Analizowany algorytm używa co najwyżej $n$ wartości funkcji z precyzją $\delta $ oraz w najgorszym przypadku ma błąd proporcjonalny do $\max{(\delta, n^{-1 / r + \varrho })}$ w klasie funkcji $\F^D_{r,\varrho }$ dla $p < \infty$ oraz w klasie $\F^C_{r,\varrho }$ dla $p \leq \infty$. Kluczowym parametrem algorytmu jest
$$
    h = T / m \quad with \quad  m \geq 2r + 1,
$$
gdzie $m$ jest początkową gęstością siatki. Dodatkowo, niech $\omega  = \omega(h)$ spełnia $0 < \omega < (r + 1)h $.

Na początku algorytm aproksymuje punkt osobliwy $s_f$. Jest to realizowane w trzech krokach. W kroku 1. przy pomocy siatki rozmiarze o długości $h$ i różnic dzielonych lokalizowany jest punkt $s_f$ na przedziale $[u_1, v_1]$ o długości $(r + 1)h$. W kroku 2. używamy wielomianów interpolujących $\tilde{p}_+$ i $\tilde{p}_-$ do zwężenia tego przedziału do $[u_2, v_2]$. Krok 3. produkuje przedział $[u_3, v_3] \subseteq [u_2, v_2]$, w którym różnica $|\tilde{p}_{+} - \tilde{p}_{-}|$ jest nierosnąca na $[u_3, \xi]$ i niemalejąca na $[\xi, v_3]$, gdzie $\xi$ jest finalną aproksymacją $s_f$.

Powyższe kroki mogą być zapisane następująco; dla $t_i = ih \; \forall i$. \vspace{10pt}

\noindent
\begin{tabular}{p{0.10\linewidth} p{0.85\linewidth}}
    
    \textit{Krok 1} & Oblicz różnice dzielone $\tilde{d}_i = \tilde{f}[t_i, \ldots, t_{i+r+1}]$ for $1 \leq i \leq m $ oraz znajdź \\
                    & \(\displaystyle \qquad i^* = arg \max_{1 \leq i \leq m }|\tilde{d}_i| \)  \\
                    & Niech $u_1 = t_{i^*}$ i $v_1 = t_{i^* + r + 1}$. \\
                    & \\

    \textit{Krok 2} & Oznaczymy przez $\tilde{p}_+$ i $\tilde{p}_-$ wielomiany stopnia $ \leq r$, które interpolują węzły $(t_j, \tilde{f}(t_j))$ odpowiednio dla $i^* - r \leq j \leq i^*$ oraz dla $i^* + r + 1 \leq j \leq i^* + 2r + 1$. Następnie wykonaj iterację: \\
                    & $u := u_1$, $v := v_1$ \\
                    & \textbf{dopóki} $v-u < \omega$ \textbf{do} \\
                    & \hspace{20pt}$z_j := u + j(v-u) / (r+2), \qquad j = 1, 2, \ldots, r + 1$ \\
                    & \hspace{20pt}\(\displaystyle j^* := arg \max_{1 \leq j \leq r + 1}|\tilde{p}_{+}(z_j) - \tilde{p}_{-}(z_j)| \) \\
                    & \hspace{20pt}\textbf{if} $|\tilde{f}(z_{j^*}) - \tilde{p}_{-}(z_j)| \leq |\tilde{f}(z_{j^*}) - \tilde{p}_{+}(z_j)|$ \textbf{then} \\
                    & \hspace{40pt}$u:= z_{j^*}$ \\
                    & \hspace{20pt}\textbf{else} \\
                    & \hspace{40pt}$v:= z_{j^*}$ \\
                    & \textbf{end while} \\
                    & Niech $u_2 = u$ i $v_2 = v$. \\
                    & \\

    \textit{Krok 3} & Wykonaj iterację: \\
                    & $u := u_2$, $v := v_2$ \\
                    & \textbf{dopóki} istnieje maksimum lokalne $|\tilde{p}_{+} - \tilde{p}_{-}|$ na $(u,v)$ \textbf{do} \\
                    & \hspace{20pt}$z :=$ największe maksimum lokalne $|\tilde{p}_{+} - \tilde{p}_{-}|$ na $(u,v)$ \\
                    & \hspace{20pt}\textbf{if} $|\tilde{f}(z) - \tilde{p}_{-}(z)| \leq |\tilde{f}(z) - \tilde{p}_{+}(z)|$ \textbf{then} \\
                    & \hspace{40pt}$u:= z$ \\
                    & \hspace{20pt}\textbf{else} \\
                    & \hspace{40pt}$v:= z$ \\
                    & \textbf{end while} \\
                    & Niech $u_3 = u$ i $v_3 = v$.
\end{tabular} \vspace{10pt}

Finalną aproksymacją $s_f$ jest
\begin{equation*}
        \xi := arg \max_{u_3 \leq x \leq v_3}|\tilde{p}_{+} - \tilde{p}_{-}| \hspace{200pt}
\end{equation*}


\mgrclosechapter


%%
%% ==== ROZDZIAŁ 4 ====
%%

\chapter{Testy numeryczne}


\mgrclosechapter



%%
%% ======== DODATKI ========
%%
% \appendix
% \chapter{----}
%%
%-> Treść dodatku A
%%
% \mgrclosechapter

%%
%% ======== BIBLIOGRAFIA ========
%%

%% <<<< BiBTeX >>>>
% \bibliography{<pliki bib>} 
%%
\begin{thebibliography}{88}

    \bibitem{IaA}
    F. Arandiga, A. Cohen, R. Donat, N. Dyn,
    \emph{Interpolation and approximation of piecewise smooth functions}, SIAM J. Numer. Anal. 43 (2005) 41–57

    \bibitem{PoA}
    L. Plaskota, G. W. Wasilkowski, Y. Zhao, 
    \emph{The power of adaption for approximating functions with singularities}, Mathematics Of Computation 77
    2008, p. 2309–2338

    \bibitem{UA}
    L. Plaskota, G. W. Wasilkowski, 
    \emph{Uniform approximation of piecewise r-smooth and globally continuous functions}, SIAM Journal on Numerical
    Analysis, Vol. 47, No. 1 (2008/2009)

    \bibitem{CoDF}
    B. Kacewicz, P. Przybyłowicz, 
    \emph{Complexity of the derivative-free solution of
    systems of IVPs with unknown singularity hypersurface}, Journal of Complexity
    
    \bibitem{AoP}
    P. M. Morkisz, L. Plaskota, 
    \emph{Approximation of piecewise Hölder functions from inexact information}, Journal of Complexity

\end{thebibliography}

\end{document}
