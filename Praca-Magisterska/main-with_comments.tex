%%% & --translate-file=cp1250pl
%% ************ AKADEMIA GÓRNICZO-HUTNICZA W KRAKOWIE **************
%% ***************** Wydział Matematyki Stosowanej ***************** 
%% ****************** PRACA MAGISTERSKA w LaTeX-u ******************
%%    autor: Tomasz Czyż
%%    Copyright (C) 2003 by ------
%% ************************* Plik główny *************************
%%
%-> Opcje klasy: 
%->  - man -> Wersja "meska" pracy :) Kwestia jednej litery w oświadczeniu. Domyślnie jest
%->           w wersji "damskiej".
%->  - robocza -> Produkuje wersję roboczą (nagłówek wzbogacony o datę kompilacji oraz
%->               nazwisko autora w stopce) 
%->  - pdftex -> Obowiązkowa przy kompilacji pdfLaTeX-em.
%->  - mfu, oik, pit, opt -> Specjalności, patrz niżej.
%->  - twoside -> standardowa w klasie mwbk. Posłuży nam do przygotowania wersji dwustronnej pracy
%%
%% ======== PREAMBUŁA ======== 
%%
\documentclass[oik, pdftex, robocza, man]{mgrwms}
%%
%-> Pakiet inputenc jeżeli TeX nie obsługuje przekodowań (pierwsza linijka pliku),
%-> a chcemy pisać z pomocą stron kodowych 
%-> (cp1250 dla Windows, iso 8859-2 (latin2) dla Linuxa)
%%
\usepackage[utf8]{inputenc}  % opcja latin2 dla Linuxa
\usepackage{amsmath}           % łatwiejszy skład matematyki
\newcommand\numberthis{\addtocounter{equation}{1}\tag{\theequation}}
\usepackage{amssymb} 
\usepackage{latexsym}
\usepackage{amsthm}
\usepackage{enumerate}

\usepackage{color}
%%
%-> ładujemy wybrany pakiet polonizacyjny.
%%
\usepackage[polish]{babel}     % pakiet 'babel' koniecznie z pakietem 'fontenc'!
\usepackage[OT4]{fontenc}
\usepackage{polski}
\allowdisplaybreaks
%\usepackage{breqn}
%%
%-> Dodatkowe polecenia które trzeba umieścić w preambule:  
%-> - \makeindex, jeżeli włączamy indeks (nieobowiązkowy)
%-> - \includeonly, dla warunkowej kompilacji, i ew. pozostałe
%%
% \makeindex
% \includeonly{---,---,---,---}
%%
%% <<<< BiBTeX >>>>
%%
% \bibliographystyle{ddabbrv}
% \nocite{*}
%%
\begin{document}
% \prefixing  % jeżeli piszemy z 'polskim' oraz używamy 'ciachów': /a, /e, /c,... 
%%
%%
%% ======== METRYCZKA PRACY ========
\title{ \LARGE Tytuł}
\author{Autor}
\promotor{dr xxx}
\nralbumu{000000}
\maketitle
%%
\slowakluczowe{słowa kluczowe}
\keywords{keywords}
%%
%-> Specjalność (tutaj nie uwzględnioną) zadajemy przez opcję klasy:
%-> (czyli np. \documentclass[man,pit]{mgrwms})
%-> - mfu (Mat. w ekonomii, finansach i ubezpieczeniach)
%-> - oik (Mat. obliczeniowa i komputerowa)
%-> - pit (Mat. w naukach technicznych i przyrodniczych)
%-> - opt (Optymalizacja) 
%%
%-> Uwaga! W przypadku niektórych, dosyć długich tytułów, ;) zasadna może być próba ZMNIEJSZENIA
%-> czcionki do stopnia \LARGE.
%-> Wykonujemy zatem: \title{\LARGE Nasz nieco przydługi temat... ;)}
%-> Jeżeli tytuł zajmuje pięć linii (lub więcej) może to okazać się wskazane dla zwiększenia 
%-> estetyki strony tytułowej.
%%
%%
%% Aby wstawić własny obrazek należy mieć go w .png, a następnie wyeksportować do (np. przy pomocy GIMPA) do .eps!
%%
%% ======== NASZE MAKRA ========
%%
%-> Miejsce na nasze makra (jedno z wielu ;). Lepszym pomysłem będzie jednak 
%-> umieszczenie ich w osobnym pliku i wczytanie poleceniem \input
%%
\newtheorem{thm}{\indent Twierdzenie}[chapter]
\newtheorem{lemma}[thm]{\indent Lemat}
\newtheorem{cor}[thm]{\indent Wniosek}
\newtheorem{obs}[thm]{\indent Obserwacja}
\newtheorem{uw}[thm]{\indent Uwaga}
\newtheorem{df}[thm]{Definicja}
\newcommand{\E}{\mathbb{E}}
\newcommand{\R}{\mathbb{R}}
\newcommand{\Pra}{\mathbb{Pra}}

\makeatletter
\newcommand*{\defeq}{\mathrel{\rlap{%
                     \raisebox{0.3ex}{$\m@th\cdot$}}%
                     \raisebox{-0.3ex}{$\m@th\cdot$}}%
                     =}
\let\c@table\c@figure
\makeatother

% \input{----}
%%
%%
%% ======== SPIS TREŚCI ========
%%
\tableofcontents
%%
%%
%% ======== STRESZCZENIE PRACY (POLSKIE) ========
%TODO streszczenie
\begin{streszczenie}
Streszczenie
 

%%
%-> Tutaj umieszczamy streszczenie po polsku (bezpośrednio lub np. ładując
%-> poleceniem \input{plik}). Jeżeli praca jest w języku polskim streszczenie
%-> NIE POWINNO przekroczyć jednej strony, jeżeli zaś w języku obcym wówczas 
%-> ma mieć MINIMUM 4 str.!
%%
%-> Treść streszczenia po polsku
%%
\end{streszczenie}
%%
%%
%% ======== STRESZCZENIE PRACY (ANGIELSKIE) ========
%TODO abstract
\begin{abstract}
Abstract

%%
%-> Podobnie jak wyżej umieszczamy abstract angielski. Sposoby umieszczenia dowolne.
%-> Tutaj też, NIE PRZEKRACZAMY jednej strony!
%%
%-> Treść streszczenia po angielsku
%%
\end{abstract}
%%
%%
%% ======== GŁÓWNA CZĘŚĆ PRACY ========
%%
%-> Wstęp/Wprowadzenie jest rozdziałem NIENUMEROWANYM. Sposób nazwania
%-> do wyboru. Domyślnie nazywa się "Wstęp", ale przez argument opcjonalny możemy to zmienić.
%-> (\begin{wstep}[Wprowadzenie]).
%%
%%
%% ==== WSTĘP ====
%%
%TODO Wstęp
\begin{wstep}    % ew. \begin{wstep}[Wprowadzenie]
Wstep

%%
%-> Treść wstępu (oczywiście może być w innym pliku...)
% \input{----}
%%
\end{wstep}
%%
%%
%% ==== ROZDZIAŁ 1 ====
%%
%%
%-> Treść rozdziału 1
%%

\chapter{Rozdział 1}

Lotem ipsum

\mgrclosechapter

%% ==== ROZDZIAŁ 2 ====
%%
%\chapter{Wyniki numeryczne}
%%
%-> Treść rozdziału 2
%%
%\mgrclosechapter
%%
%-> W podobny sposób umieszczamy kolejne rozdziały, pamiętając by umieścić 
%-> na końcu polecenie \mgrcloschapter (!)
%%
%%
%% ======== DODATKI ========
%%
%-> Po rozdziałach miejsce na ewentualne dodatki. 
% \appendix
% \chapter{----}
%%
%-> Treść dodatku A
%%
% \mgrclosechapter
%%
%%
%% ======== BIBLIOGRAFIA ========
%%
%-> OBOWIĄZKOWA część pracy. Umieszczamy zaraz po ostatnim rozdziale.  
%-> Możemy użyć środowiska thebibliography lub też bazy BiBTeX.
%-> Sposób formatowania linii z daną pozycją literatury, 
%-> umieszczony jest w pliku mgrinfo.pdf (katalog doc\mgrinfo)
%-> 
%% <<<< BiBTeX >>>>
%%
% \bibliography{<pliki bib>} 
%%
\begin{thebibliography}{88}
%%
%-> Treść bibliografii. 
% \bibitem{<Klucz>} ... 
%%
\end{thebibliography}

\end{document}

%% =========================================================== %%